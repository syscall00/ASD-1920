\documentclass{article}
\usepackage{graphicx}
\usepackage[italian]{babel}
\usepackage{pdfpages}
\usepackage[utf8]{inputenc}
\usepackage[a4paper, total={6.5in, 8in}]{geometry}
\usepackage{listings}
\usepackage{hyperref}
\usepackage{comment}
\usepackage{amsmath}
\usepackage{caption}
\usepackage{multirow}
\usepackage{mathtools}


\makeatletter
% \@seccntformat is the command that adds the number to section titles
% we make it a no-op
\renewcommand{\@seccntformat}[1]{}
\makeatother



\title{Relazione Algoritmi e Strutture Dati 2019/20} 
\author{Vannella Alessio}
\date{}
\begin{document}
\maketitle
\tableofcontents
\begin{abstract}
Questa breve relazione spiega le scelte progettuali attuate per la realizzazione del progetto didattico del corso di Algoritmi e Strutture dati dell'università di Torino nell'anno didattico 2019/2020. 
\end{abstract}
\vfill
\subsubsection*{Note sui tempi misurati e sulla macchina utilizzata}
Le misurazioni sono fatte su una macchina con le seguenti caratteristiche:
\begin{itemize}
    \item Processore: Intel Core i5-1035G1 1.00GHz;
    \item Memoria principale : 24 GB;
    \item Sistema operativo: Parrot 4.8;
\end{itemize}
I tempi nei programmi implementati in C sono calcolati attraverso la funzione clock() della libreria "time.h", che restituisce il tempo impiegato dalla cpu ad eseguire il processo. Per questo motivo, i tempi di attesa effettivi potrebbero essere diversi rispetto a quelli calcolati tramite questa funzione.\newline\newline
I tempi nei programmi implementati in Java, invece, sono calcolati attraverso il metodo currentTimeMillis del package Java.lang.System che restituisce il timestamp del sistema al momento della chiamata. Quindi non si tratta del tempo di esecuzione del processo in cpu, ma del tempo effettivo atteso dall'utente.\newline\newline I tempi sono tutti misurati in secondi.
\newpage

\input{Chap1}

\newpage

\section{Esercizio 2 - Edit distance}
È richiesta l'implementazione di un algoritmo che calcoli la distanza tra due stringhe s1 ed s2 prese in input. In particolare viene richiesta la realizzazione di due algoritmi. Entrambi gli algoritmi si basano su queste osservazioni:
\begin{itemize}
    \item se $|s1| = 0$, allora edit$\_$distance$(s1, s2) = |s2|$;
    \item se $|s2| = 0$, allora edit$\_$distance$(s1, s2) = |s1|$;
    \item altrimenti, siano:
    \begin{itemize}
        \item $d_{no-op}: \begin{cases} \text{edit\_distance}(rest(s1), rest(s2)) & \text{se } s_1[0] = s_2[0]\\$ 
        $\infty & \text{altrimenti } \end{cases}$

        \item $d_{canc}$: 1 + edit$\_$distance$(s1, rest(s2))$
        \item $d_{ins}$: 1 + edit$\_$distance$(rest(s1), s2)$
    \end{itemize}
\end{itemize}
Si ha: edit$\_$distance$(s1, s2) = min\{d_{no-op}, d_{canc}, d_{ins}\}$



\subsection{Edit distance naive}
Il primo algoritmo viene implementato ricorsivamente e segue in pieno le osservazioni descritte sopra. L'Implementazione in java è autoesplicativa.\newline Questa tipo di implementazione è ingenua;  calcola volta per volta l'edit distance di ogni sottostringa e questo la rende molto costoso dal punta di vista temporale: il limite superiore di questa soluzione è $O(k^n)$.

\subsection{Edit distance con Dynamic Programming - Memoization}
Proprio per la natura dell'algoritmo precedente, si può pensare ad un miglioramento attraverso l'uso della programmazione dinamica, che permette in caso siano presenti sottoproblemi ricorrenti, di velocizzare significativamente l'algoritmo.
In particolare il secondo algoritmo utilizza la tecnica di memoization.\newline Infatti dopo un'analisi del grafo di esecuzione dell'algoritmo precedente, si può notare che ci sono molti valori di edit distance calcolati più volte. \newline Il secondo algoritmo sfruta proprio questo fatto e ogni volta che calcola salva, volta per volta, l'edit distance della sottosequenza calcolata in una matrice di dimensione $n \cdot m$ (con n ed m lunghezza delle due stringhe) e ritorna l'edit distance presente nell'ultima cella della matrice. Questa soluzione è asintoticamente ottima e ha tempo $O(n \cdot m)$.


\begin{table}[h]
\centering
\begin{tabular}{|c|c|c|c|c|c|}
\hline
\textbf{}       & \textbf{' '} & \textbf{v} & \textbf{vi} & \textbf{vin} & \textbf{vino} \\ \hline
\textbf{' '}    & 0            & 1          & 2           & 3            & 4             \\ \hline
\textbf{v}      & 1            & 0          & 1           & 2            & 3             \\ \hline
\textbf{vi}     & 2            & 1          & 0           & 1            & 2             \\ \hline
\textbf{vin}    & 3            & 2          & 1           & 0            & 1             \\ \hline
\textbf{vina}   & 4            & 3          & 2           & 1            & 2             \\ \hline
\textbf{vinai}  & 5            & 4          & 3           & 2            & 3             \\ \hline
\textbf{vinaio} & 6            & 5          & 4           & 3            & \textbf{2}    \\ \hline
\end{tabular}
\caption*{Matrice di edit distance calcolata con DP}

\end{table}


\subsection{Considerazioni finali}
Il tempo di esecuzione della seconda versione dell'algoritmo è notevolmente più veloce rispetto alla prima versione.\newline Dall'esecuzione dell'applicazione, abbiamo che il tempo di calcolo dell'edit distance per il dataset è di circa 23 secondi.\newline Un'eventuale esecuzione sullo stesso dataset con la versione senza memoization dell'algoritmo richiederebbe diversi ordini di tempo in più.
 
\newpage
\section{Esercizio 3 - Confronto tra Hash table e array statico }
In questo esercizio viene richiesta l'implementazione della struttura dati hashmap in modo che possa accettare tipi di dati generici. \newline Inoltre è richiesta la misurazione dei tempi di caricamento di un dataset di prova nella struttura dati definita e in un array statico e la misurazione del get di 10.000.000 chiavi generate casualmente da entrambe le strutture dati.
\subsection{Implementazione}
La struttura dati implementata è un hashmap nella quale le collisioni sono gestite tramite concatenazione. In particolare, è definito un tipo di dato Node che permette di implementare una lista concatenata per ogni indice della tabella. La funzione di hash utilizzata è un semplice modulo sul numero di elementi della tabella.

\subsection{Dati raccolti}
L'array è un array statico di 6.321.078 elementi. Viene riordinato utilizzando l'algoritmo di quicksort che ha complessità temporale $\Theta(nlogn)$ nel caso medio, e i dati vengono recuperati attraverso l'algoritmo di ricerca binaria, sfruttando l'ordinamento dell'array.\newline\newline
L'hash table ha una tabella di 6.321.078 elementi. Utilizza una funzione di hash modulo limitata dalla dimensione della tabella.\newline
Gli elementi sono generati con un seme statico basato su un contatore, in modo da rendere la distribuzione delle chiavi e dei valori casuali, ma uguali ad ogni esecuzione.
\begin{table}[h]
\centering
\begin{tabular}{|c|c|c|c|c|}
\hline
\textbf{}               & \textbf{Elementi presenti}        & \textbf{Elementi raccolti} & \textbf{T caricamento} & \textbf{T recupero} \\ \hline
\textbf{Array ordinato} & \multirow{2}{*}{\textbf{6321078}} & \textbf{6321929}                  & \textbf{1.852}                & \textbf{4.950}             \\ \cline{1-1} \cline{3-5} 
\textbf{Hash table}     &                                   & \textbf{6321929}                  & \textbf{1.910}                & \textbf{1.406}             \\ \hline
\end{tabular}
\end{table}


\end{document}

