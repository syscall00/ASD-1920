\section{Esercizio 3 - Confronto tra Hash table e array statico }
In questo esercizio viene richiesta l'implementazione della struttura dati hashmap in modo che possa accettare tipi di dati generici. \newline Inoltre è richiesta la misurazione dei tempi di caricamento di un dataset di prova nella struttura dati definita e in un array statico e la misurazione del get di 10.000.000 chiavi generate casualmente da entrambe le strutture dati.
\subsection{Implementazione}
La struttura dati implementata è un hashmap nella quale le collisioni sono gestite tramite concatenazione. In particolare, è definito un tipo di dato Node che permette di implementare una lista concatenata per ogni indice della tabella. La funzione di hash utilizzata è un semplice modulo sul numero di elementi della tabella.

\subsection{Dati raccolti}
L'array è un array statico di 6.321.078 elementi. Viene riordinato utilizzando l'algoritmo di quicksort che ha complessità temporale $\Theta(nlogn)$ nel caso medio, e i dati vengono recuperati attraverso l'algoritmo di ricerca binaria, sfruttando l'ordinamento dell'array.\newline\newline
L'hash table ha una tabella di 6.321.078 elementi. Utilizza una funzione di hash modulo limitata dalla dimensione della tabella.\newline
Gli elementi sono generati con un seme statico basato su un contatore, in modo da rendere la distribuzione delle chiavi e dei valori casuali, ma uguali ad ogni esecuzione.
\begin{table}[h]
\centering
\begin{tabular}{|c|c|c|c|c|}
\hline
\textbf{}               & \textbf{Elementi presenti}        & \textbf{Elementi raccolti} & \textbf{T caricamento} & \textbf{T recupero} \\ \hline
\textbf{Array ordinato} & \multirow{2}{*}{\textbf{6321078}} & \textbf{6321929}                  & \textbf{1.852}                & \textbf{4.950}             \\ \cline{1-1} \cline{3-5} 
\textbf{Hash table}     &                                   & \textbf{6321929}                  & \textbf{1.910}                & \textbf{1.406}             \\ \hline
\end{tabular}
\end{table}
